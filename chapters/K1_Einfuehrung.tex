\chapter{Einführung} % Main chapter title

\label{K1} % For referencing the chapter elsewhere, use \ref{K1} 

%----------------------------------------------------------------------------------------

% Define some commands to keep the formatting separated from the content 


%----------------------------------------------------------------------------------------

Auf den folgenden Seiten wird die Motivation für dieses Promotionsprojekt dargelegt und das Themengebiet anfänglich situiert. Weiterhin ist eine Kapitelübersicht enthalten sowie einzelne Hinweise zur stilistischen und formellen Ausgestaltung dieser Schrift.

%---------------------------------------------------------------


\section{Ausgangslage}
\label{K1:sec:ausgangslage}\largerpage

%---------------------------------------------------------------

Seit ungefähr 2010 lassen sich innerhalb des ohnehin exponentiell wachsenden Forschungsgebietes der digitalen Datenverarbeitung noch stärker herausragende Innovationen und Leistungszugewinne beobachten. Das autonome Fahren scheint plötzlich in greifbare Nähe gerückt zu sein, Technik und Technologie sind so kostengünstig und vielfältig wie nie zuvor, oder kosten für den Endnutzer gar nichts mehr dank engagierter Organisationen, die qualitativ hochwertige Open Data- und Open Source-Projekte betreuen. Dank Smartphones und vergleichbarer mobiler Endgeräte sind Milliarden Menschen mit dem Internet verbunden und können hierüber Informationen zu nahezu jedem Lebensbereich abrufen und beitragen. Kurz gefasst: Allein innerhalb der vergangenen zehn Jahre hat sich auf diesem Gebiet sowohl strukturell, finanziell, ideologisch, innovativ als auch konzeptuell derart viel getan, dass diese Aufzählung nur die gröbste aller denkbaren Einbettungen liefern kann.

All dies ist nur auf Grundlage der Akkumulation von immensen Datenmengen, internationalen Wissensverbünden und dem weltweiten Technologie- sowie Informationstransfer möglich. Die hier vorgestellte Monographie mit dem Titel \emph{\ttitle{}} greift dabei einen Teilaspekt aus dem Blickwinkel der Sprachtechnologieforschung, Übersetzungswissenschaft sowie Kommunikationswissenschaft auf. Anhand des Skype Translators als Fallbeispiel wird der Frage nachgegangen, welche Auswirkung die Verbindung von computervermittelter Kommunikation und Maschineller Übersetzung (MÜ) auf das Nutzungsverhalten derartiger Technologien hat. Nicht erst innerhalb des genannten Zehnjahreszeitraums hat sich, auch geleitet vom exponentiellen Wachstum der Rechenfähigkeit und der immer effizienter werdenden Computertechnologie, das Feld der Mensch-Maschine-Interaktion ausgeweitet. Eine Unterform davon ist die sog. Usability, sprich: die Nutzerfreundlichkeit, bei der die Interaktion von Mensch und Maschine in Hinblick auf den erfolgreichen Austausch von Informationen hin untersucht wird. Hieran anschließend lautet die Frage, wie eine Kommunikationssituation unter Beteiligung der MÜ strukturiert wird.

Diese zwei Leitfragen lassen sich ferner unterteilen in kleinere Einheiten auf den Gebieten der Soziolinguistik und der Sprachtechnologie. Welche Eigenschaften weist ein Chat mit Beteiligung eines Echtzeit-Übersetzungsdienstes auf? Inwiefern kann in diesem Kontext von einem Gespräch die Rede sein? Welche Konsequenzen hat es zuletzt für vergleichsweise kleine Sprachen im europäischen Kontext wie dem Katalanischen, wenn eine solche Technologie zum Einsatz kommt? Wie formt -- auf der vorausgehenden Frage aufbauend -- der Einsatz von solchen Technologien wie dem Skype Translator die Informationsextraktion aus Textchats?

Über all diesen Fragen schwebt zudem unausgesprochen die allgemeine Sorge der gesamten Sprachmittlerbranche, in wenigen Jahren nicht mehr benötigt zu werden. Bereits häufig bemüht und sicherlich auch mittlerweile als Verweis abgenutzt ist dabei der Babelfish aus Douglas Adams' \emph{Per Anhalter durch die Galaxis}. Der Grund, weshalb dieser kleine, fiktive Fisch, den man sich zwecks Universalübersetzungswerkzeug aller Sprachen in alle Sprachen ins Ohr stecken kann, auch in dieser Arbeit genannt wird, ist das nach wie vor sehnsüchtig betrachtete Menschheitsziel, womöglich irgendwann Sprachgrenzen mithilfe von Maschinen komplett überwindbar und somit obsolet zu machen. Diese Sehnsucht steht in einem Widerspruch zur exotischen Anziehung von Fremdsprachen, die wiederum auch einen Kontrast bildet zur Hegemonie des Englischen in der globalisierten Welt, und dort besonders im Internet. 

%---------------------------------------------------------------

\section{Hintergrund}
\label{K1:sec:Hintergrund}\largerpage

%---------------------------------------------------------------

Das übergeordnete Themengebiet des Forschungsvorhabens ist die Translationstechnologie, konkreter: die Maschinelle Übersetzung (MÜ) sowie die automatische Verarbeitung natürlicher Sprache. Da mit Katalanisch jedoch auch -- im europäischen Kontext -- eine vergleichsweise \glqq kleine\grqq{} Sprache im Fokus der Ausarbeitung steht, gewinnt die Monographie einen soziolinguistischen Anteil hinzu. Zwar ist es nicht Ziel dieser Arbeit, auf Grundlage der Betrachtung des Skype Translators eine Abwägung der Vor- und Nachteile sowie die Chancen und Risiken für kleinere Sprachen zu disktuieren. Allerdings besteht der Anspruch, einen möglichen Einstieg für die Auseinandersetzung aufzuzeigen und einige Denkanstöße auch für weitere Forschung auf diesem Gebiet zu liefern.

Die Kommunikationssoftware Skype\is{Microsoft!Skype} ist bereits seit 2003 auf dem Markt. Ursprünglich als eigenständiges Unternehmen unter gleichem Namen gegründet, wurde es 2011 von Microsoft aufgekauft. Weltweit bekannt ist das Programm vor allem für die Möglichkeit, Echtzeit-Videochats zu führen. Auch wenn es eine explizit beworbene Business-Version für Unternehmen und Dienstleister gibt, wird Skype hauptsächlich von Privatanwendern genutzt. Vereinzelt lassen sich jedoch auch gerade Dienstleister verzeichnen, die die Videofunktion zur Fernberatung ihrer Kunden einsetzen (s. z.\,B.\ Leipziger Wohnungsbaugesellschaft (LWB)). Gegenwärtig ist Skype neben dem ebenfalls aus den USA stammenden Konkurrenten \emph{Zoom}\is{Kommunikationstechnologie!Zoom}\is{Zoom} auch medial präsent. Im Zuge der durch die weltweite Corona-Pandemie bestehenden Kontakt- und Reisebeschränkungen und dem entsprechend sich verändernden Kommunikationsverhalten kommt es vor, dass ein offizielles Interview im Fernsehen per Videochat durchgeführt wird. Skype ist dabei eine häufig genutzte Anwendung.

Die hier zur Diskussion gestellte Funktion des Skype Translators wurde 2015 implementiert und befindet sich im stetigen Wandel. Bereits die Version, die dieser Arbeit zugrunde liegt, ist eine andere, als die aktuell für die Endnutzer·innen verfügbare. Zu Beginn der Ausarbeitung war sie nur für Windows 7 und 10 verfügbar und auch dort nur in der jeweils aktuellsten Version der Software. Gegenwärtig jedoch ist der Skype Translator bei Skype auf Windows, Linux und MacOS sowie in der Smartphone-App enthalten. Die Version \emph{Skype for Business} unterstützt den Skype Translator bislang hingegen nicht. Aktuell sind mehr als 60 Sprachen im Textchat sowie 11 Sprachen für die Übersetzung von gesprochener Sprache in Voice- und Videochats bereitgestellt. Dieser Bereich befindet sich in einem steten Wandel und wird im Jahresrhythmus von Neuerungen geprägt. Die Darstellung bestehender MÜ-Ansätze, von regelbasierter über statistischer bis hin zur jüngsten neuronalen MÜ, ist in diesem Zusammenhang notwendig, um die Funktionalität des Skype Translators einschätzen zu können und belastbare Hypothesen für diese Arbeit aufzustellen.

Da es mittlerweile eine Vielzahl weltweit agierender Dienste gibt, die ihren Nutzer·innen eine maschinelle Übersetzungsausgabe zur Verfügung stellen (z.\,B.\ Facebook, Instagram\is{Instagram|see{Facebook}})\is{Facebook!Instagram}, wird der Skype Translator zunächst in den gegenwärtigen Forschungsstand der MÜ eingebettet. Hierzu ist in der Arbeit ein Abriss über die derzeit verfügbaren Technologien enthalten, bevor auch Computer Assisted Translation (CAT) und Computer Assisted Interpretation (CAI) Tools betrachtet werden, um den Skype Translator dazwischen anzusiedeln. Weiterhin wird im theoretischen Teil dieses Buches auf Evaluierungsmöglichkeiten der MÜ eingegangen.

Eine besondere Berücksichtigung erfährt dabei das Post-Editing (PE), da die traditionsreiche, jedoch zugleich sich stetig weiterentwickelnde Technologie wie die MÜ bei Weitem nicht fehlerfrei arbeitet und daher der Nachbesserung durch den Menschen bedarf. Innerhalb des Forschungsstrangs zum Post-Editing findet sich eine Vielzahl an Publikationen zu Evaluations- und Qualitätssicherungsmethoden, die in diesem Zusammenhang abstrahiert und auf das Modell angepasst werden können. Diese Punkte stecken die Ecken des Untersuchungsfeldes ab, in dem die Informationsverarbeitung in mehrsprachigen Textchats praktisch untersucht wird.

Der o.\,g.\ soziolinguistische Anteil wird parallel dazu ausgearbeitet. Das Katalanische als untersuchte Sprache ist vor allem durch die Unabhängigkeitsbewegung in Katalonien bekannt. Mit ca. 9 Mio. aktiven Sprechern ist das Katalanische, das in seinen Varietäten, die in Katalonien, der Valencianischen Gemeinschaft, Südfrankreich, den Balearen, Andorra und einem kleinen Teil von Sardinien angesiedelt sind, annähernd so groß wie die Gesamtbevölkerung Belgiens oder Tschechiens. Das Katalanische ist jedoch keine offizielle Amtssprache der EU. Außerdem ist es einzig in Andorra alleinige offizielle Sprache. In Spanien haben die Regionalsprachen laut Verfassung den Status einer offiziellen Sprache innerhalb der jeweiligen autonomen Gemeinschaften inne, wobei das Katalanische durch den Regionalstatut der autonomen Gemeinschaft gestützt wird \citep[414]{bochmann_sprachpolitik_2011}. Aufgrund des zentralistischen Staatsaufbaus in Frankreich gestaltet sich die Situation sprachpolitisch dort anders.

Diese knappe Skizze stellt die Notwendigkeit der soziolinguistischen Betrachtung innerhalb des Forschungsvorhabens dar, wobei der Schwerpunkt hier besonders auf die Stellung der Sprache im digitalen Raum und damit wieder auf dem Informationsaustausch in mehrsprachigen Konstellationen liegt. Von dort aus lässt sich wiederum die Brücke schlagen zur Sprachtechnologie und dem Einsatz der maschinellen Übersetzung.

Der empirisch-experimentelle Teil ist in drei größere Bereiche gegliedert. Neben einer grundlegenden Skizzierung des Nutzungsverhaltens von Skype und dem Skype Translator auf Grundlage eines Online-Fragebogens, sieht der zweite Teil eine Eye-Tracking-Studie vor, in der deutschsprachige Studierende während einer Chatkommunikation mit katalanischen Muttersprachler·innen vom Skype Translator unterstützt werden. Der dritte Teil, ebenfalls eine Eye-Tracking-Studie, liefert hierzu Vergleichsmaterial. Studierende werden während der Chatkommunikation mit anderen deutschsprachigen Muttersprachler·innen erfasst, ohne dass der Skype Translator aktiviert ist.

Zuletzt werden, ausgehend von den gewonnen Erkenntnissen am Skype Translator, weiterführende Überlegungen für die Forschung abstrahiert.


%---------------------------------------------------------------

\section{Kapitelübersicht}
\label{K1:sec:gliederung}\largerpage

%---------------------------------------------------------------

Wie bereits in der Einführung deutlich wird, setzt diese Arbeit an mehreren unterschiedlichen Themengebieten und Teilaspekten an. Um hieraus eine logische Ordnung mit angemessener Argumentationsstruktur aufzubauen, setzt das folgende Kapitel~\ref{K2} zunächst ganz allgemein in der Linguistik an. So wird eine Definition der computervermittelten Kommunikation herausgearbeitet (Abschnitt~\ref{K2:sec:CMC}, S.\,\pageref{K2:sec:CMC}), die sowohl der Wissensrepräsentation (Abschnitt~\ref{K2:sec:Wissensrepräsentation}, S.\,\pageref{K2:sec:Wissensrepräsentation}) als auch einen Textbegriff im Rahmen der digitalen Lebenswelt (Abschnitt~\ref{K2:sec:Textbegriff}, S.\,\pageref{K2:sec:Textbegriff}) des Menschen, und damit besonders dem Chat, Rechnung trägt. Die Rückanbindung an die in dieser Arbeit beteiligten Sprachen findet ebenfalls in diesem Kapitel ab Abschnitt~\ref{K2:sec:Status-Sprache-DigRaum} (S.\,\pageref{K2:sec:Status-Sprache-DigRaum}) statt.

Kapitel~\ref{K3} greift die theoretischen Grundlagen auf und verknüpft sie mit einem geschichtlichen Überblick über die Entwicklung von Sprachtechnologien (Abschnitt~\ref{K3:sec:Geschichte-Technologie}, S.\,\pageref{K3:sec:Geschichte-Technologie}). Die historische Übersicht endet mit dem gegenwärtigen Stand der Forschung zur MÜ (Abschnitt~\ref{K3:sec:MUE}, S.\,\pageref{K3:sec:MUE}), in dessen Rahmen eine Differenzierung der Technologien (Abschnitt~\ref{K3:sec:Dienste}, S.\,\pageref{K3:sec:Dienste}) stattfindet. Skype als Herzstück dieser Arbeit erhält dabei eine besondere Zuwendung (Abschnitt~\ref{K3:subsec:Skype}, S.\,\pageref{K3:subsec:Skype}). Weiterhin umfasst das Kapitel die Qualitätsbewertung, sowohl mit Blick auf das Leistungsvermögen der Technologie (Abschnitt~\ref{K3:sec:Qualitaet}, S.\,\pageref{K3:sec:Qualitaet}) als auch aus Sicht der Endnutzer·innen in Form der \emph{Usability} (Abschnitt~\ref{K3:sec:Usability}, S.\,\pageref{K3:sec:Usability}). Kapitel \ref{K4} (S.\,\pageref{K4}) leitet mit der Aufstellung der konkreten Forschungsfragen vom Theorieteil zum praktischen Teil dieser Arbeit über.

Kapitel~\ref{K5} dient dann der Darstellung der methodologischen Grundlagen. Nach einer generellen Vorstellung der Gepflogenheiten bei der Arbeit mit On\-line-Um\-fra\-gen und mit Eye-Tracking-Studien (Abschnitt~\ref{K5:sec:Methodik}, S.\,\pageref{K5:sec:Methodik}) erfolgt die Konzeption beider Erhebungsformate: Für die Online-Umfrage in Abschnitt~\ref{K5:subsec:Konzept-Feldstudie} (S.\,\pageref{K5:subsec:Konzept-Feldstudie}) und für die Feldstudie mit dem Skype Translator in Abschnitt~\ref{K5:sec:Feldstudie} (S.\,\pageref{K5:sec:Feldstudie}). 

Die beiden Kapitel~\ref{K6} und \ref{K7} umfassen dann die Präsentation, Analyse und Diskussion der gewonnenen Daten. Zunächst werden die Ergebnisse der Online-Umfrage (Abschnitt~\ref{K6:sec:UmfrageAllg}, S.\,\pageref{K6:sec:UmfrageAllg}) vorgestellt und situiert. Anschließend erfolgt die Beschreibung der beiden Studienvarianten \emph{Katalanisch-Deutsch} (Abschnitt~\ref{K6:sec:Probandinnen-CatDe}, S.\,\pageref{K6:sec:Probandinnen-CatDe}) und \emph{Deutsch-Deutsch} (Abschnitt~\ref{K6:sec:Probandinnen-DeDe}, S.\,\pageref{K6:sec:Probandinnen-DeDe}), bevor die Daten dieser beiden Teile ebenfalls analysiert werden (Abschnitte~\ref{K6:sec:DatenCatDe}, S.\,\pageref{K6:sec:DatenCatDe} und \ref{K6:sec:DatenDeDe}, S.\,\pageref{K6:sec:DatenDeDe}). Kapitel~\ref{K7} dient der zusammenführenden Diskussion der Ergebnisse sowohl aus visueller als auch statistischer Sicht (Abschnitte~\ref{K7:sec:visuelle-inspektion}, S.\,\pageref{K7:sec:visuelle-inspektion} und \ref{K7:sec:fixations}, S.\,\pageref{K7:sec:fixations}) und verweist darüber hinaus auf Bereiche dieser Arbeit, die lohnenswert für eine eingehende Folgeuntersuchung erscheinen (\ref{K7:sec:BlindeFlecke}, S.\,\pageref{K7:sec:BlindeFlecke}).

Mit Kapitel~\ref{K8} (S.\,\pageref{K8}) wird ein Fazit gezogen und bestehende Desiderata ausgesprochen.

%---------------------------------------------------------------

\section{Formelle Hinweise}
\label{K1:sec:Formalia}

%---------------------------------------------------------------
\begin{sloppypar}
Zum Abschluss der Einleitung seien an dieser Stelle noch einzelne Hinweise formeller Natur genannt. Die Monographie ist gemäß der Stilleitfäden von Language Science Press formatiert und ausgestalltet worden. Die sog. \emph{Generic Style Rules for Linguistics}\footnote{S. hierzu \url{https://langsci.github.io/gsr/GenericStyleRulesLangsci.pdf}. Letzter Zugriff am \datum{}.} können ebenso online eingesehen werden wie die \emph{Language Science Press Guidelines}\footnote{S. hierzu \url{https://langsci.github.io/guidelines/latexguidelines/LangSci-guidelines.pdf}. Letzter Zugriff am \datum{}.}. Beispiele, erstmalige Erwähnungen, fremdsprachliche Ausdrücke sowie nicht konventionalisierte Abkürzungen werden kursiv geschrieben. Zitate, die kürzer als drei Zeilen sind, werden in den folgenden Anführungszeichen gesetzt: \glqq \ldots \grqq{}\end{sloppypar}

Über die Arbeit hinweg werden textinterne Referenzen wie Tabellen, Beispiele und Abbildungen immer mit der fortlaufenden Nummerierung sowie der Seitenanzahl angegeben. Eine Ausnahme wird nur gemacht, wenn eine unmittelbar anliegende Textpassage direkt auf das Element verweist, dann erscheint nur die jeweilige fortlaufende Nummer.

% Zahlen von 1 bis 12 werden gemäß einer aktuellen Verlautbarung des Duden nicht mehr ausgeschrieben\footnote{Siehe hierzu \url{https://www.duden.de/sprachwissen/sprachratgeber/Schreibung-von-Zahlen-0}, letzter Aufruf am \datum{}.}.

%\begin{sloppypar}Unterkapitel, die inhaltlich zwar eine abgeschlossene Einheit darstellen, zugleich jedoch größere Themenwechsel vollziehen, werden zur besseren Leitung des Leseflusses mit einem vorangestellten, fett geschriebenen Schlagwort eingeleitet. Beispielsweise umfasst das Analysekapitel zu Regressionen (Abschnitt~\ref{K6:Subsubsec:regression:catde}, S.\,\pageref{K6:subsubsec:regpd:catde}) sowohl die Betrachtung der eingehenden als auch ausgehenden Regressionen. Der jeweilige Anfang der entsprechenden Abschnitte ist dementsprechend gekennzeichnet.\end{sloppypar}

Diese Arbeit umfasst zudem zwei Indices. Der erste Index ist ein allgemeines Stichwortverzeichnis mit wichtigen Begriffen dieser Arbeit. Der zweite Index ist ein erweitertes Namensregister. Er führt die Namen wichtiger Personen im Umfeld dieses Forschungsbereichs auf.
